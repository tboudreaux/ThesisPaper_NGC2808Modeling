\documentclass[twocolumn,linenumbers]{src/aastex631}
\newcommand{\vdag}{(v)^\dagger}
\newcommand\aastex{AAS\TeX}
\newcommand\latex{La\TeX}
\newcommand\fidanka{\texttt{Fidanka} }

\usepackage{amsmath}
\usepackage{cancel}
\usepackage{float}

\shorttitle{Self Consistently Modeling NGC 2808}
\shortauthors{Boudreaux et al.}
% \watermark{DRAFT}
\graphicspath{{./}{figures/}{src/figures}}

\begin{document}

\title{Chemically Self-Consistant Modeling of the Globular Cluster NGC 2808 and its Effects on the Inferred Helium abundance of Multiple Stellar Populations.}

\correspondingauthor{Emily M. Boudreaux}
\email{emily.m.boudreaux.gr@dartmouth.edu, emily@boudreauxmail.com}

\author[0000-0002-2600-7513]{Emily M. Boudreaux}
\affiliation{Department of Physics and Astronomy, Dartmouth College, Hanover, NH 03755, USA}

\author[0000-0003-3096-4161]{Brian C. Chaboyer}
\affiliation{Department of Physics and Astronomy, Dartmouth College, Hanover, NH 03755, USA}

\author{Amanda Ash}
\affiliation{Department of Physics and Astronomy, University of North Georgia, Dahlonega, GA 30533, USA}

\author{Renata Edaes Hoh}
\affiliation{Department of Physics and Astronomy, Dartmouth College, Hanover, NH 03755, USA}


\author[0000-0002-2012-7215]{Gregory Feiden}
\affiliation{Department of Physics and Astronomy, University of North Georgia, Dahlonega, GA 30533, USA}


\begin{abstract}
  The Helium abundances in the multiple populations which are now known to
  comprise all closely studied Milky Way globular clusters are often inferred
  by fitting isochohrones generated from stellar evolutionary models to
  globular cluster photometry. It is therefore important to build stellar
  models that are chemically self-consistent in terms of their structure,
  atmosphere, and opacity. In this work we present the first chemically
  self-consistent stellar models of the Milky Way Globular Cluster NGC 2808
  using MARCS model atmospheres, OPLIB high-temperature radiative opacities,
  and AESOPUS low-temperature radiative opacities. These stellar models were
  fit to the NGC 2808 photometry using \fidanka, a new software tool that was
  developed optimally fit cluster photometry to isochrones and for population
  synthesis.  \fidanka can determine, in a relatively unbiased way, the ideal
  number of distinct populations which exist within a dataset and then fits
  isochrones to each population. We achieve this through a combination of
  Bayesian Gaussian Mixture Modeling and a novel number density estimation
  algorithm. Using \fidanka and F275W-F814W photometry from the Hubble UV
  Globular Cluster Survey we find that the helium abundance of the second
  generation of stars in NGC 2808 is higher than the first generation by
  $15\pm3\%$. This is in agreement with previous studies of NGC 2808. {\bf This
  work, along with previous work by \citet{Dotter2016} focused on NGC 6752
  demonstrates that chemically self-consistent models of globular clusters do
  not signifigantly alter infered helium abundances and are therefor unlikeley
  to be worth the signifigant additional time investment.}
\end{abstract}
% \begin{abstract}
% 	Globular Clusters (GCs) provide a unique astrophysical laboratory for
% 	studying the formation and evolution of stars. GCs are old, dense, and it
% 	has historically been believed that they have a single stellar population. However, in the
% 	last two decades, it has been definitively shown that most if not all Milky
% 	Way GCs have multiple stellar populations (MPs). These MPs are chemically
% 	distinct from one another, primarily separated by light element, including helium, abundance
% 	variations without the standard accompanying heavy element abundance
% 	variations. As the precise formation channel of these MPs remains
% 	an open question, and one which is sensitive to the population - population
% 	compositional differences, the extent of the composition
% 	variations between MPs is a key parameter to constrain. Many metal abundances may be
% 	directly measured spectroscopically; however, helium abundances are not
% 	directly observable in GCs. Instead, helium abundances are inferred from
% 	stellar models. It is therefore important to understand build stellar
% 	models that are self-consistent in the compositions of the structure,
% 	atmosphere, and opacity. In this work we present the first chemically
% 	self-consistent stellar models of the Milky Way Globular Cluster NGC 2808 using MARCS model atmospheres, OPLIB high-temperature radiative opacities, and AESOPUS low-temperature radiative opacities.
% 	We find that the helium abundance of the second generation of stars is
% 	higher than the first generation by {\color{red} SOME AMOUNT}.
% 	{\color{blue} This is in agreement with previous studies of NGC 2808.}
% \end{abstract}

\keywords{Globular Clusters (656), Stellar evolutionary models (2046)}

\section{Introduction}\label{sec:Intro}
Introduction


\section{Chemical Consistency}\label{sec:const}
There are three primary areas in which must the stellar models must be
consistent: the atmospheric boundary conditions, the opacities, and interior
abundances. The interior abundances are relatively easily handled by adjusting
parameters within our stellar evolutionary code. However, the other two areas are
more complicated to bring into consistency. Atmospheric boundary conditions and
opacities must both be calculated with a consistent set of chemical abundances
outside of DSEP. The

\subsection{Atmospheric Boundary Conditions}\label{sec:atm}
Certain assumptions {\color{red}[WHICH ONES,CITATION]} made in stellar structure
codes, such as DSEP, are valid when the optical depth of a star is small.
However, in the atmospheres of stars, the number density of particles drops low
enough and the optical depth consequently becomes large enough that these
assumptions break down, and more separate plasma modeling code is required.
Generally structure code will use tabulated atmospheric boundary conditions
generated by these specialized codes {\color{red} PHEONIX, ATLAS, MARCS +
[CITATIONS]}. Often, as the boundary conditions are both expensive to compute
and not the speciality of stellar structure researchers, the boundary
conditions are not updated as as light-element interior abundance varies. 

One key element when chemically consistently modeling NGC 2808 modeling is the
incorporation of new atmospheric models with the same elemental abundances as
the structure code. We use atmospheres generated from the \texttt{MARCS} grid of
model atmospheres \citep{Plez2008}. \texttt{MARCS} provides one-dimensional,
hydrostatic, plane-parallel and spherical LTE atmospheric models
\citep{Gustafsson2008}. Model atmospheres are made to match the
spectroscopically measured elemental abundances of populations A and E.
Moreover, for each populations, atmospheres with  various helium mass fractions
are generated. These range from Y=0.24 to Y=0.36 in steps of 0.02. A comparison
of the pressure and temperature throughout the atmospheres of the two
populations with helium abundances representative of literature values is shown
in Figure \ref{fig:AEAtmComp}.

\begin{figure}
	\centering
	\includegraphics[width=0.5\textwidth]{src/figures/notebookFigures/AtmosphereComparison.pdf}
	\label{fig:AEAtmComp}
	\caption{Comparison of the MARCS model atmospheres generated for the two
	extreme populations of NGC 2808. These lines shows population A and E with
	the same Helium abundance; though, we fit a grid of models over various
	helumn abundances. Dashed lines show the temperature of the boundary
	condition while sold lines show the pressure.}
\end{figure}


\subsection{Opacities}\label{sec:opac}
In addition to the atmospheric boundary conditions, both the high and low
temperature opacities used by DSEP must be made chemically consistent. Here we
use OPLIB high temperature opacity tables \citep{Colgan2016} retrieved using
the TOPS web-interface. Low temperature opacity tables are retrieved from the
Aesopus 2.0 web-interface \citep{Marigo2009, Marigo2022}. Ideally, these
opacities would be the same used in the atmospheric models. However, the
opacities used in the MARCS models are not publicly available. As such, we use
the opacities provided by the TOPS and Aesopus 2.0 web-interfaces.


\section{Modeling}\label{sec:modeling}
One key element of NGC 2808 modeling is the incorporation of new atmospheric
models, generated from the \texttt{MARCS} grid of model atmospheres \citep{Plez2008},
which match interior elemental abundances. \texttt{MARCS} provides one-dimensional,
hydrostatic, plane-parallel and spherical LTE atmopsheric models
\citep{Gustafsson2008}. Members of our collaboration have generated atmospheric 
models for populations A and E.  Integration of these new model atmospheres
into DSEP is ongoing. 

For similar reasons as discussed in \S\ref{sec:p1} we conduct this research
with OPLIB high-temperature opacity tables as opposed to OPAL tables. We will
also generate low temperature opacity tables using the \texttt{MARCS}.
Moreover, we confirm that the atmosphere and structure meet in an optically
thick region of the star by shifting the atmospheric fitting point from an optical
depth of $\tau = 2/3$ (used by DSEP currently for \texttt{PHOENIX} model atmospheres) to
some higher $\tau$. We will experiment to identify the best optical depth to
fit at.. 

These population have been studied in depth by Feiden and their chemical
compositions were determined in \citet{Milone2015} (see Table 2 in that paper).
While we cannot yet evolve DSEP models with these new boundary conditions, we
can make a first pass investigation of the affect of OPLIB opacities (Figure
\ref{fig:NGC2808ISO}). Note how the models generated using OPLIB opacity tables
have a systematically lower luminosity. This discrepancy is consistent with
the overall lower opacities of the OPLIB tables. 


The isochrones generally used to infer the degree of helium enhancements assume that
convection operates in the same manner in metal-poor stars as it does in the
Sun. However, observations from \textit{Kepler} of metal-poor red giants
\citep{Bonaca2012, tayar2017correlation}, in concert with interferometric
radius determination of the metal-poor sub-giant HD 140283
\citep{creevey2015benchmark}, have shown that the efficiency of convection
changes with iron content. As the final portion of our work to more carefully
handle a star's chemistry, we will modify DSEP to capture this variation in
convective efficiency. 


\section{fidanka}\label{sec:fidanka}
When fitting isochrones to the data we have four main critera for any method

\begin{itemize}
	\item The method must be robust enough to work along the entire main sequence, turn off, and much of the subgian and red giant branchs.
	\item Any method should consider photometric uncertanity in the fitting process.
	\item The method should be model independent, weighting any n number of populations equally.
	\item The method should be automated and require minimal intervention from the user.
\end{itemize}


While there are many packages which can measure fiducial lines very well
{\color{red}[CITATIONS]}, we do not belive that any of these perfectly match
our use case. Therefore, we elect to develope our own software suite,
\fidanka. \fidanka is a python package designed to automate
much of the process of measuring fiducual lines in CMDs, adhearing to the four
criteria we lay out above. Primary features of \fidanka may be seperated into
two categories: stellar population synthethis and isochrone
optimization/fitting (distance modulus, B-V color excess, and binpary mass
fraction)

\subsection{Fiducual Line Measurment}
\fidanka takes a interative approach to measuring fiducial lines, the first
step of which is to make a ``guess'' as to the fiducial line. This guess will
be used to verticalize the CMD so that further alorithms can work in 1-D
magnitude bins without worrying about weighting issues caused by varying
projections of the evolutionary sequence onto the magnitude axis. This initial
guess is calculated by splitting the CMD into magnitude bins, with uniform
numbers of stars per bin (so that bins are cover a small magnitude range over
densly populated regions of the CMD while covering a much larger mangnitude
range in sparsly populated regions of the CMD, such as the RGB). A unimodal
gaussian distribution is then fit to the color distribution of each bin, and
the resulting mean color is used as thec initial fiducial line guess. This
rough fiducual line will approximatly trace the area of highest density. We
subtract the color of this fiducual line from that of each star to verticalize
the CMD. 

If \fidanka were to simply apply the same algorithm to the verticalized CMD
then the resulting fiducial line would simply be a re-extraction of the initial
fiducial line. To avoid this, we take a more robust, number density based
approach, which considers the distribution of stars in both color and magnitude
space simultaniously. For each star in the CMD we first using a
\texttt{introselect} partitioning algorithm to select the 50th nearest stars.
To account for the case where the star is at an extreme edge of the CMD, those
50 stars include the star itself (such that we really select 49 stars + 1). We
use \texttt{qhull}\footnote{https://www.qhull.com}\citep{Barber1996, } to
calculate the convex hull of those 50 points. The number density at each star
then is defined as 50/(area of convex hull). Because we use a fixed number of
points per star, and a partitioning algorithm as opposed to a sorting
algorithm, this method scales like $\mathcal{O}(n)$, where n is the number of
stars in the CMD. It also intrinsically weights the density of of each star
equally as the counting statistics of per convex are uniform. We are left with a
CMD where each star has a defined number density {\color{red}[FIGURE]}.

We can adapt this density map method to consider photometric uncertanties by
adopting a simple monte carlo approach. For each star in the CMD we draw a
random magnitude shift from a gaussian distribution with a standard deviation
equal to the photometric uncertanty of the star in each filter. We then shift
each star by this random amount and calculate the density map as before. We
repeat this process $m$ times, and then take the median density at each point
in the CMD {\color{red}[FIGURE]}. This method will result in noisy density
peaks with charectaristic width less than that of the photometric noise
smoothing out; while wider peaks will remain.

\fidanka can now exploit this density map to fit a better fiducual line to the
data, as the densiy map is far more robust to outlies. There are multiple
algorithms we impliment to fit the fiducial line to the color density profile
in each magnitude bin; they are explained in more detail in the \fidanka
documentation. However, of most relevance here is the A* path finding heuristic
over peaks. Peaks are extracted from the The color-density profile in at
density bin {\color{red}[FIGURE]}. The A* heuristic is then used to find an
optimal path (though not neccesarily {\em the} optimal path) through the peaks
{\color{red}[FIGURE]}. The resulting path is then used as the fiducial line
{\color{red}[FIGURE]}. The heursitc considers the slope of the path from one
bin to the next and simultaniously the change in density from one bin to the
next. This allows the path to be robust to both steep slopes and low density
regions. Due to the heuristic the fiducial line path  is not neccesarily the
vertical line defined by the initial fiducial line guess and will instead be a
more optimized/better guess.

This method of fiducial line extraction is very effective at tracing the
meandian ridge line of the overall CMD; however, it struggles to discriminate
between multiple populations. The density variations from one population to the
next are often too subtle. Moreover, when sampling the density. Moreover, the
spacing between main sequence populations may be of a similar order to the
photmetric uncertantities, and therefor the indivudual populations may smear
into one and other. For these reasons, \fidanka does not attempt to extract
multiple unique fiducuial lines for each population; instead, it measures the
width of the overall sequence. Width measurment again makes use of the
color-density pofile, selecting defining the width as the difference in color
between the 5th and 95th percentile of the density {\color{red}[FIGURE]}.

\subsection{Stellar Population Synthethis}
In addition to measuring fiducial lines, \fidanka also includes a stellar
population synthethis module. This module is used to generate synthetic CMDs
from a given set of isochrones. This is of primary importance for binary
population modelling. The module is also used to generate synthetic CMDs for
the purpose of testing the fiducial line extraction algorithms against priors.

\fidanka uses MIST formated isochrones {\color{red}[CITATION]} as input along
with distance modulus, B-V color excess, binary mass fraction, and bolometric
corrections. An arbitrarily large number of isochrones may be used to define an
arbitrary number of populations. Synthetic stars are samples from each
isochroner based on a definiable probability (for example it is belived that
$\sim90\%$ of stars in globular clusters are younger population
{\color{red}[CITATION]). Based on the metallicity, $\mu$, and E(B-V) of each
isochrone, bolometric corrections are taken from bolometric correction
tables. Where bolometric correction tables do not include exact metallicities
or extinctions a linear interpolation is preformed between the two bounding
values. {\color{red}[FIGURE]} shows an example of a synthetic CMD generated
from a set of 2 NGC 2808 isochrones as well as a comparison between those isochrones and
the measured fiducual line of the synthetic population.




\section{Isochrone Fitting}\label{sec:isoFit}
We fit pairs of isochrones to the HUGS data for NGC 2808 using \fidanka, as
descrbed in \S {\color{red}FIDANKA SECTION}. Two isochrones, one for Population
A and one for Population E are fit simultaneously. These isochrones are
constrained to have the same distance modulus, $\mu$, and color excess, E(B-V).
Moreover, we constrain the mixing length, $\alpha_{ML}$, to be constant between
any two isochrones in a set. For every isochrone in the set of combination of
which fulfilling these constraints $\mu$, $E(B-V)$, Age$_{A}$, and $Age_{B}$
are optimized to reduce the $\chi^{2}$ distance between the fiducial
line{\color{red}s} and the isochrones. Because we fit fiducial lines directly,
we do not need to consider the binary population fraction, $f_{bin}$, as a free
parameter.

{\color{red}Table \ref{tab:isoFit}} shows the results of the isochrone fitting
for NGC 2808. The best fit isochrones are shown in {\color{red}Figure
\ref{fig:isoFit}.}

{\color{red} Disscus the implications of the specific isochrones we fit.}

\subsection{ACS-HUGS Photometric Zero Point Offset}
The Hubble legacy archive photometry used in this work is calibrated to the
Vega magnitude system. However, we have found that the photometry has a
systematic offset of {\color{red}$\sim0.02$ magnitudes} in the F814W band when
compared to the same stars in the ACS survey {\color{red} [CITATION]}. The
exact cause of this offset is unknown, but it is likely due to a difference in
the photometric zero point between the two surveys. A full correction of this
offset would require a careful re-reduction of the HUGS photometry, which is
beyond the scope of this work. We instead recognize a 0.02 inherent uncertainty
in the inferred magnitude of any fit when comparing to the ACS survey. This
uncertainty is {\color{red} small when compared to the uncertainty in the
distance modulus and should not affect the conclusion of this
paper}.

{\color{red} Figure Showing the offset}

{\color{blue}The independent analysis of the Hubble Space Telescope photometry
ACS and HUGS, and an additional ground-based dataset created by Stetson et al.
(2009), of the globular cluster NGC2808 resulted in findings of a systematic
difference in the magnitude of the stars. A significant offset between the
three datasets was found. HUGS and ACS were found to have consistent data for
Vvega, however an offset of approximately 0.025 was found in the Ivega. ACS and
Stetson were found to have inconsistent differences in both Vvega and Ivega.
The three photometric studies do not match, and future studies which compare
theoretical models to the data will need to take into account these systematic
uncertainties. }


% \section{Results}\label{sec:results}
Using \fidanka we fit pairs of Population A + E isochrones to the HUGS data for
NGC 2808. Each pair of isochrones is allowed to vary in distance modulus,
reddening, relative helium mass fraction (A/E), and age. The $\chi^{2}$
distribution for the isochrone pairs is shown in Figure {\color{red}[FIGURE]}.
The best fit isochrones are shown in Figure {\color{red}[FIGURE]} and optimized
parameters for these are presented in Table {\color{red}[TABLE]}.


% \input{src/sections/discussion.tex}

\section{Conclusion}\label{sec:conclusion}
{\color{red} Here we have preformed the first chemically self-consistnent
modeling of the Milky Way Globular Cluster NGC 2808. We find that, updated
atmospheric boundary conditions and opacity tables do not have a significant
effect on the inferred helium abundances of multiple populations.}



\begin{acknowledgments}
	This work has made use of the NASA astrophysical data system (ADS). We
	would like to thank Elisabeth Newton and Aaron Dotter for their support and
	for useful disscusion related to the topic of this paper. Additionally, we
	would like to thank Kara Fagerstrom, Aylin Garcia Soto, and Keighley
	Rockcliffe for their useful disscusion related to in this work. We
	acknowledge the support of a NASA grant (No. 80NSSC18K0634). 
\end{acknowledgments}

% \software{
% 	The Dartmouth Stellar Evolution Program (DSEP) \citep{Dotter2008},
% 	\texttt{FreeEOS} \citep{Irwin2012},
% 	\texttt{pyTOPSScrape} \citep{Boudreaux22}
% }


\bibliography{src/bib/ms}{}
\bibliographystyle{aasjournal}


\end{document}
