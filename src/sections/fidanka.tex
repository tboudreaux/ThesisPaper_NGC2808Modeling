\section{fidanka}\label{sec:fidanka}
When fitting isochrones to the data we have four main critera for any method

\begin{itemize}
	\item The method must be robust enough to work along the entire main sequence, turn off, and much of the subgian and red giant branchs.
	\item Any method should consider photometric uncertanity in the fitting process.
	\item The method should be model independent, weighting any n number of populations equally.
	\item The method should be automated and require minimal intervention from the user.
\end{itemize}


While there are many packages which can measure fiducial lines very well
{\color{red}[CITATIONS]}, we do not belive that any of these perfectly match
our use case. Therefore, we elect to develope our own software suite, \fidanka.
\fidanka is a python package designed to automate much of the process of
measuring fiducual lines in CMDs, adhearing to the four criteria we lay out
above. Primary features of \fidanka may be seperated into three primary
categories: fiducual line measurment, stellar population synthethis, and
isochrone optimization/fitting. Additionally, there are utility functions which
are detailed in the \fidanka documentation.

\subsection{Fiducual Line Measurment}
\fidanka takes a interative approach to measuring fiducial lines, the first
step of which is to make a ``guess'' as to the fiducial line. This guess will
be used to verticalize the CMD so that further alorithms can work in 1-D
magnitude bins without worrying about weighting issues caused by varying
projections of the evolutionary sequence onto the magnitude axis. This initial
guess is calculated by splitting the CMD into magnitude bins, with uniform
numbers of stars per bin (so that bins are cover a small magnitude range over
densly populated regions of the CMD while covering a much larger mangnitude
range in sparsly populated regions of the CMD, such as the RGB). A unimodal
gaussian distribution is then fit to the color distribution of each bin, and
the resulting mean color is used as thec initial fiducial line guess. This
rough fiducual line will approximatly trace the area of highest density. We
subtract the color of this fiducual line from that of each star to verticalize
the CMD. 

If \fidanka were to simply apply the same algorithm to the verticalized CMD
then the resulting fiducial line would simply be a re-extraction of the initial
fiducial line. To avoid this, we take a more robust, number density based
approach, which considers the distribution of stars in both color and magnitude
space simultaniously. For each star in the CMD we first using a
\texttt{introselect} partitioning algorithm to select the 50th nearest stars.
To account for the case where the star is at an extreme edge of the CMD, those
50 stars include the star itself (such that we really select 49 stars + 1). We
use \texttt{qhull}\footnote{https://www.qhull.com}\citep{Barber1996, } to
calculate the convex hull of those 50 points. The number density at each star
then is defined as 50/(area of convex hull). Because we use a fixed number of
points per star, and a partitioning algorithm as opposed to a sorting
algorithm, this method scales like $\mathcal{O}(n)$, where n is the number of
stars in the CMD. It also intrinsically weights the density of of each star
equally as the counting statistics of per convex are uniform. We are left with a
CMD where each star has a defined number density {\color{red}[FIGURE]}.

We can adapt this density map method to consider photometric uncertanties by
adopting a simple monte carlo approach. For each star in the CMD we draw a
random magnitude shift from a gaussian distribution with a standard deviation
equal to the photometric uncertanty of the star in each filter. We then shift
each star by this random amount and calculate the density map as before. We
repeat this process $m$ times, and then take the median density at each point
in the CMD {\color{red}[FIGURE]}. This method will result in noisy density
peaks with charectaristic width less than that of the photometric noise
smoothing out; while wider peaks will remain.

\fidanka can now exploit this density map to fit a better fiducual line to the
data, as the densiy map is far more robust to outlies. There are multiple
algorithms we impliment to fit the fiducial line to the color density profile
in each magnitude bin; they are explained in more detail in the \fidanka
documentation. However, of most relevance here is the A* path finding heuristic
over peaks. Peaks are extracted from the The color-density profile in at
density bin {\color{red}[FIGURE]}. The A* heuristic is then used to find an
optimal path (though not neccesarily {\em the} optimal path) through the peaks
{\color{red}[FIGURE]}. The resulting path is then used as the fiducial line
{\color{red}[FIGURE]}. The heursitc considers the slope of the path from one
bin to the next and simultaniously the change in density from one bin to the
next. This allows the path to be robust to both steep slopes and low density
regions. Due to the heuristic the fiducial line path  is not neccesarily the
vertical line defined by the initial fiducial line guess and will instead be a
more optimized/better guess.

This method of fiducial line extraction is very effective at tracing the
meandian ridge line of the overall CMD; however, it struggles to discriminate
between multiple populations. The density variations from one population to the
next are often too subtle. Moreover, when sampling the density. Moreover, the
spacing between main sequence populations may be of a similar order to the
photmetric uncertantities, and therefor the indivudual populations may smear
into one and other. For these reasons, \fidanka does not attempt to extract
multiple unique fiducuial lines for each population; instead, it measures the
width of the overall sequence. Width measurment again makes use of the
color-density pofile, selecting defining the width as the difference in color
between the 5th and 95th percentile of the density {\color{red}[FIGURE]}.

\subsection{Stellar Population Synthethis}
In addition to measuring fiducial lines, \fidanka also includes a stellar
population synthethis module. This module is used to generate synthetic CMDs
from a given set of isochrones. This is of primary importance for binary
population modelling. The module is also used to generate synthetic CMDs for
the purpose of testing the fiducial line extraction algorithms against priors.

\fidanka uses MIST formated isochrones {\color{red}[CITATION]} as input along
with distance modulus, B-V color excess, binary mass fraction, and bolometric
corrections. An arbitrarily large number of isochrones may be used to define an
arbitrary number of populations. Synthetic stars are samples from each
isochroner based on a definiable probability (for example it is belived that
$\sim90\%$ of stars in globular clusters are younger population
{\color{red}[CITATION]}). Based on the metallicity, $\mu$, and E(B-V) of each
isochrone, bolometric corrections are taken from bolometric correction tables.
Where bolometric correction tables do not include exact metallicities or
extinctions a linear interpolation is preformed between the two bounding
values. {\color{red}[FIGURE]} shows an example of a synthetic CMD generated
from a set of 2 NGC 2808 isochrones as well as a comparison between those
isochrones and the measured fiducual line of the synthetic population.

\subsection{Isochrone Optimization}
The optimization routines in \fidanka will find the best fit distance modulus,
B-V color excess, and binary number fraction for a given set of isochrones. If
a single isochrone is provided then the optimization is done by mimizing the
$\chi^2$ of the perpendicular distnaces between an isochrone and a fiducual
line. If multiple isochrones are provided then those isochrones are first used
to run stellar population synthethis and generate a synthetic CMD. The
optimization is then done by mimizing the $\chi^2$ of both the perpendicular
distnaces between and widths of the observed fiducual line and the fiducual
line of the synthetic CMD.


