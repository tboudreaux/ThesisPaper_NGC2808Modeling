\section{Isochrone Fitting}\label{sec:isoFit}
We fit pairs of isochrones to the HUGS data for NGC 2808 using \fidanka, as
descrbed in \S {\color{red}FIDANKA SECTION}. Two isochrones, one for Population
A and one for Population E are fit simultaneously. These isochrones are
constrained to have the same distance modulus, $\mu$, and color excess, E(B-V).
Moreover, we constrain the mixing length, $\alpha_{ML}$, to be constant between
any two isochrones in a set. For every isochrone in the set of combination of
which fulfilling these constraints $\mu$, $E(B-V)$, Age$_{A}$, and $Age_{B}$
are optimized to reduce the $\chi^{2}$ distance between the fiducial
line{\color{red}s} and the isochrones. Because we fit fiducial lines directly,
we do not need to consider the binary population fraction, $f_{bin}$, as a free
parameter.

{\color{red}Table \ref{tab:isoFit}} shows the results of the isochrone fitting
for NGC 2808. The best fit isochrones are shown in {\color{red}Figure
\ref{fig:isoFit}.}

{\color{red} Disscus the implications of the specific isochrones we fit.}

\subsection{ACS-HUGS Photometric Zero Point Offset}
The Hubble legacy archive photometry used in this work is calibrated to the
Vega magnitude system. However, we have found that the photometry has a
systematic offset of {\color{red}$\sim0.02$ magnitudes} in the F814W band when
compared to the same stars in the ACS survey {\color{red} [CITATION]}. The
exact cause of this offset is unknown, but it is likely due to a difference in
the photometric zero point between the two surveys. A full correction of this
offset would require a careful re-reduction of the HUGS photometry, which is
beyond the scope of this work. We instead recognize a 0.02 inherent uncertainty
in the inferred magnitude of any fit when comparing to the ACS survey. This
uncertainty is {\color{red} small when compared to the uncertainty in the
distance modulus and should not affect the conclusion of this
paper}.

{\color{red} Figure Showing the offset}

{\color{blue}The independent analysis of the Hubble Space Telescope photometry
ACS and HUGS, and an additional ground-based dataset created by Stetson et al.
(2009), of the globular cluster NGC2808 resulted in findings of a systematic
difference in the magnitude of the stars. A significant offset between the
three datasets was found. HUGS and ACS were found to have consistent data for
Vvega, however an offset of approximately 0.025 was found in the Ivega. ACS and
Stetson were found to have inconsistent differences in both Vvega and Ivega.
The three photometric studies do not match, and future studies which compare
theoretical models to the data will need to take into account these systematic
uncertainties. }
