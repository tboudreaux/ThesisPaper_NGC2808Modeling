\section{Chemical Consistency}\label{sec:const}
There are three primary areas in which must the stellar models must be made
chemically consistent: the atmospheric boundary conditions, the opacities, and
interior abundances. The interior abundances are relatively easily handled by
adjusting parameters within our stellar evolutionary code. However, the other
two areas are more complicated to bring into consistency. Atmospheric boundary
conditions and opacities must both be calculated with a consistent set of
chemical abundances outside of the stellar evolution code. For evolution we use
the Dartmouth Stellar Evolution Program (DSEP) \citep{Dotter2008}, a well
tested 1D stellar evolution code which has a particular focus on modelling low
mass stars ($\le 2$ M$_{\odot}$)

\subsection{Atmospheric Boundary Conditions}\label{sec:atm}
Certain assumptions, primarily that the radiation field is at equilibrium and
radiative transport is diffusive \citep{Salaris2005}, made in stellar structure
codes, such as DSEP, are valid when the optical depth of a star is large.
However, in the atmospheres of stars, the number density of particles drops low
enough and the optical depth consequently becomes small enough that these
assumptions break down, and separate, more physically motivated, plasma
modeling code is required. Generally structure code will use tabulated
atmospheric boundary conditions generated by these specialized codes, such as ATLAS9
\citep{Kurucz1993}, PHOENIX \citep{Husser2013}, MARCS \citep{Gustafsson2008},
and MPS-ATLAS \citep{Kostogryz2023}. Often, as the boundary conditions are
expensive to compute, they are not updated as interior abundances vary. 

One key element when chemically consistently modeling NGC 2808 modeling is the
incorporation of new atmospheric models with the same elemental abundances as
the structure code. We use atmospheres generated from the \texttt{MARCS} grid
of model atmospheres \citep{Plez2008}. \texttt{MARCS} provides one-dimensional,
hydrostatic, plane-parallel and spherical LTE atmospheric models
\citep{Gustafsson2008}. Model atmospheres are made to match the
spectroscopically measured elemental abundances of populations A and E.
Moreover, for each population, atmospheres with various helium mass fractions
are generated. These range from Y=0.24 to Y=0.36 in steps of 0.03. All
atmospheric models are computed to an optical depth of $\tau = 100$ where their
temperature and pressures serves as boundary conditions for the structure code.
In general, enhancing helium in the atmosphere has only a small impact on the atmospheric
temperature profile, while leading to a drop in the pressure by $\sim 10 - 20 \%$.

\subsection{Opacities}\label{sec:opac}
In addition to the atmospheric boundary conditions, both the high and low
temperature opacities used by DSEP must be made chemically consistent. Here we
use OPLIB high temperature opacity tables \citep{Colgan2016} retrieved using
the TOPS web-interface. Retrival of High termperature opacities is done using
\texttt{pyTOPSScrape}, first introduced in \citet{Boudreaux2023}. Low
temperature opacity tables are retrieved from the Aesopus 2.0 web-interface
\citep{Marigo2009, Marigo2022}. Ideally, these opacities would be the same used
in the atmospheric models. However, the opacities used in the MARCS models are
not publicly available. As such, we use the opacities provided by the TOPS and
Aesopus 2.0 web-interfaces.
