\section{Introduction}\label{sec:Intro}
Globular clusters (GCs) are among the oldest observable objects in the
universe \citep{Pen11}. They are characterized by high densities with typical
half-light radii of $\le$10 pc \citep{Vanderburg2010}, and typical masses
ranging from $10^{4}$--$10^{5}$ M$_{\odot}$ \citep{Bro06} --- though some GCs
are significantly larger than these typical values {\color{red}EXAMPLE}. GCs
provide a unique way to probe stellar evolution \citep{Bau03}, galaxy formation
models \citep{Boy18,Kra05}, and dark matter halo structure \citep{Hud18}.
{\color{red} BRING IN SOME MORE RECENT CITATIONS.}

The traditional view of Globular Clusters was, for a long time, that they
consisted of a single stellar population (SSP, in some publications this is
referred to as a Simple Stellar Population). This view was supported by
spectroscopically uniform heavy element abundances \citep{Carretta2010, Bastian2018} accross most clusters (M54 and $\omega$Cen are notable exceptions, see \citet{Marino2015} for further details), and the lack of evidence for multiple stellar populations
(MPs) in past color-magnitude diagrams of GCs \citep[i.e.][]{Sandage1953, Alcaino1975}. However, over the last 40 years non-trivial star-to-star light-element abundance variations have been observed \citep[i.e.][]{Smith1987} and, in
the last two decades, it has been definitively shown that most if not all Milky
Way GCs have MPs \citep{Gratton2004, Gratton2012, Piotto2015}. The lack of photometric evidence for MPs
can be attributed to the short color throw available to ground based
photometric surveys \citep{Milone2017}; specifically, lacking UV filters. While MPs are chemically distinct
from one another, that distinction is most prominent when observing with $U$
and $B$ filters \citep{Sbordone2011}.

The prevalence of multiple populations in GCs is so distinct that the proposed
definitions for what constitutes a globular cluster now often center the
existence of MPs. Whereas, people have have often tried to categorized objects
as GCs through relations between half-light radius, density, and surface
brightness profile, in fact many objects which are generally thought of as GCs
don't cleanly fit into these cuts \citep{Peebles1968, Brown1991, Brown1995, Bekki2002}.
Consequently, \citet{Carretta2010} proposed a definition of GC based on
observed chemical inhomogeneities in their stellar populations. The modern
understanding of GCs then is not simply one of a dense cluster of stars which
may have chemical inhomogeneities and multiple populations; rather, it is one
where those chemical inhomogeneities and multiple populations themselves are
the defining element of a GC.

All Milky Way globular clusters older than 2 Gyr studied in detail show
populations enriched in He, N, and Na while also being deplete in O and C
\citep{Piotto2015,Bastian2018}. These light element abundance patterns also are
not strongly correlated with variations in heavy element abundance, resulting
in spectroscopically uniform Fe abundances between populations. Further,
high-resolution spectral studies reveal anti-correlations between N-C
abundances, Na-O abundances, and potentially Al-Mg \citep{Sneden1992,
Gratton2012}. Typical stellar fusion reactions can deplete core oxygen;
however, the observed abundances of Na, Al, and Mg cannot be explained by the
likes of the CNO cycle \citep{Prantzos2007}.

Formation channels for these multiple populations remain a point of debate
among astronomers. Most proposed formation channels consist of some older,
more massive, population of stars polluting the pristine cluster media before a
second population forms, now enriched in heavier elements which they themselves could
not have generated \citep[for a detailed review see ][]{Gratton2012}. The four
primary candidates for these polluters are asymptotic giant branch stars
\citep[AGBs,][]{Ventura2001,DErcole2010}, fast rotating massive stars
\citep[FRMSs,][]{Decressin2007}, super massive stars
\citep[SMSs,][]{Denissenkov2014}, and massive interacting binaries
\citep[MIBs,][]{deMink2009, Bastian2018}. 

Hot hydrogen burning (proton capture), material transport to the surface, and
material ejection into the intra-cluster media are features of each of these
models and consequently they can all be made to {\it qualitatively} agree with
the observed elemental abundances. However, none of the standard models can
currently account for all specific abundances \citep{Gratton2012}. AGB and FRMS
models are the most promising; however, both models have difficulty reproducing
severe O depletion \citep{Ventura2009,Decressin2007}. Moreover, AGB and FRMS
models require significant mass loss ($\sim 90\%$) between cluster formation
and the current epoch --- implying that a significant fraction of halo stars
formed in GCs \citep{Renzini2008,DErcole2008,Bastian2015}.

In addition to the light-element anti-correlations observed it is also known
that younger populations are significantly enhanced in Helium
\citep{Piotto2007, Piotto2015, Latour2019}. Depending on the cluster, Helium
mass fractions as high as $Y=0.4$ have been inferred \citep[e.g][]{Milone2015}.
However, due to the relatively high and tight temperature range of partial
ionization for He it cannot be observed in globular clusters; consequently, the
evidence for enhanced He in GCs originates from comparison of theoretical
stellar isochrones to the observed color-magnitude-diagrams of globular
clusters. Therefore, a careful handling of chemistry is essential when modeling
with the aim of discriminating between MPs; yet, only a very limited number of
GCs have yet been studied with chemically self-consistent (structure and
atmosphere) isochrones \citep[e.g.][NGC 6752]{Dotter2015}. 


{\color{blue} NGC 2808 is the prototype globular cluster to host Multiple
Populations.} Various studies since 2007 have identified that it may host
anywhere from 2-5 stellar populations. These populations have been identified
both spectroscopically \citep[i.e.][]{} and photometrically \citep[i.e.][]{}.
Note that recent work \citep{Valle2022} calls into question the statistical significance
of the detections of more than 2 populations in the spectroscopic data. Here we
present new, chemically self-consistent modeling of the photometry of the two extreme populations
of NGC 2808 identified by \citet{Milone2015}, populations A and E.
{\color{blue} Additionally, we present a likelihood analysis of the photometric
data of NGC 2808 to determine the number of populations present in the cluster.}


