\section{Stellar Models}\label{sec:modeling}
We use the Dartmouth Stellar Evolution Program \citep[DSEP, ][]{Dotter2008} to
generate stellar models. DSEP is a one-dimensional stellar
evolution code which includes a mixing length model of convection,
gravitational settling, and diffusion. Using the solar composition presented in
\citep{Grevesse2007} (GAS07), MARCS model atmosphers, OPLIB high temperature
opacities, and AESOPUS 2.0 low temperautre opacities we find a solar calibrated
mixing length parameter, $\alpha_{MLT}$, of $\alpha_{MLT} = 1.901$.

We use DSEP to evolve stellar models ranging in mass from 0.3 to 2.0 solar
masses from the fully convective pre-main sequence to the tip of the red giant
branch. Below 0.7 $M_{\odot}$ we evolve a model every 0.03 $M_{\odot}$ and
above 0.7 $M_{\odot}$ we evolve a model every 0.05 $M_{\odot}$. We evolve
models over a grid of mixing length parameters from $\alpha_{MLT} = 1.0$ to
$\alpha_{MLT} = 2.0$ in steps of 0.1. For each mixing length, a grid of models
and isochrones were calculated, with chemical compositions consistent with
Populations A and E (see Tables \ref{tab:comp} and \ref{tab:simpleComp}) and a range of helium abundances
(Y=0.24, 0.27, 0.30, 0.33, 0.36, and 0.39). In total,144 sets of isochrones,
each with a unique composition and mixing length were calculated. Each model is
evolved in DSEP with typical numeric tolerences of one part in $10^{7}$. Each
model is allowed a maximum time step of 50 Myr. 

\begin{deluxetable}{c|cc||c|cc}\label{tab:comp}

\tablecaption{Population Composition}

\tablenum{1}

  \tablehead{\colhead{Element} & \colhead{Pop A} & \colhead{Pop E} & \colhead{Element} & \colhead{Pop A} & \colhead{Pop E} 
} 

\startdata
Li & -0.08 & --- & In & -1.46 & --- \\
Be & 0.25 & --- & Sn & -0.22 & --- \\
B & 1.57 & --- & Sb & -1.25 & --- \\
C & 6.87 & 5.91 & Te & -0.08 & --- \\
N & 6.42 & 6.69 & I & -0.71 & --- \\
O & 7.87 & 6.91 & Xe & -0.02 & --- \\
F & 3.43 & --- & Cs & -1.18 & --- \\
Ne & 7.12 & 6.7 & Ba & 1.05 & --- \\
Na & 5.11 & 5.7 & La & -0.03 & --- \\
Mg & 6.86 & 6.42 & Ce & 0.45 & --- \\
Al & 5.21 & 6.61 & Pr & -1.54 & --- \\
Si & 6.65 & 6.77 & Nd & 0.29 & --- \\
P & 4.28 & --- & Pm & -99.0 & --- \\
S & 6.31 & 5.89 & Sm & -1.3 & --- \\
Cl & -1.13 & 4.37 & Eu & -0.61 & --- \\
Ar & 5.59 & 5.17 & Gd & -1.19 & --- \\
K & 3.9 & --- & Tb & -1.96 & --- \\
Ca & 5.21 & --- & Dy & -1.16 & --- \\
Sc & 2.02 & --- & Ho & -1.78 & --- \\
Ti & 3.82 & --- & Er & -1.34 & --- \\
V & 2.8 & --- & Tm & -2.16 & --- \\
Cr & 4.51 & --- & Yb & -1.42 & --- \\
Mn & 4.3 & --- & Lu & -2.16 & --- \\
Fe & 6.37 & --- & Hf & -1.41 & --- \\
Co & 3.86 & --- & Ta & -2.38 & --- \\
Ni & 5.09 & --- & W & -1.41 & --- \\
Cu & 3.06 & --- & Re & -2.0 & --- \\
Zn & 2.3 & --- & Os & -0.86 & --- \\
Ga & 0.78 & --- & Ir & -0.88 & --- \\
Ge & 1.39 & --- & Pt & -0.64 & --- \\
As & 0.04 & --- & Au & -1.34 & --- \\
Se & 1.08 & --- & Hg & -1.09 & --- \\
Br & 0.28 & --- & Tl & -1.36 & --- \\
Kr & 0.99 & --- & Pb & -0.51 & --- \\
Rb & 0.26 & --- & Bi & -1.61 & --- \\
Sr & 0.61 & --- & Po & -99.0 & --- \\
Y & 1.08 & --- & At & -99.0 & --- \\
Zr & 1.45 & --- & Rn & -99.0 & --- \\
Nb & -0.8 & --- & Fr & -99.0 & --- \\
Mo & -0.38 & --- & Ra & -99.0 & --- \\
Tc & -99.0 & --- & Ac & -99.0 & --- \\
Ru & -0.51 & --- & Th & -2.2 & --- \\
Rh & -1.35 & --- & Pa & -99.0 & --- \\
Pd & -0.69 & --- & U & -2.8 & --- \\
\enddata
\tablecomments{Relative Metal composition used where a(H) = 12. Where the relative composition is the the same for both populations A and E it is only listed in the population A colum for the sake of visual clarity.}
\tablerefs{\citet{Milone2015}}
\end{deluxetable}

\begin{deluxetable*}{c|c c c c c c c c c c c}\label{tab:simpleComp}

\tablecaption{Population Abundance Ratios}

\tablenum{2}

  \tablehead{\colhead{Population} & \colhead{[Fe/H]} & \colhead{[$\alpha$/Fe]} & \colhead{[C/Fe]} & \colhead{[N/Fe]} & \colhead{[O/Fe]} & \colhead{[r/Fe]} & \colhead{[s/Fe]} & \colhead{C/O} & \colhead{X} & \colhead{Y} & \colhead{Z} 
} 

\startdata
  A & -1.13 & 0.32 & -0.43 & -0.28 & 0.31 & -1.13 & -1.13 & 0.10 & 0.7285 & 0.2700 & 0.00154 \\
  E & -1.13 & -0.11 & -1.39 & -0.02 & -0.66 & -1.13 & -1.13 & 0.10 & 0.7594 & 0.240 & 0.00063
\enddata
\tablecomments{Abundance Ratios for populations A and E in NGC 2808.}
\tablerefs{\citet{Milone2015}}
\end{deluxetable*}

For each combination of population, $Y$, and $\alpha_{MLT}$ we use the
isochrone generation code first presented in \citet{Dotter2016} to generate a
grid of isochrones. The isochrone generation code identified equivalent
evolutionary points (EEPs) over a series of masses and interpolates between
them. The grid of isochrones generated for this work is avalible as a digital
supplement to this paper \dataset[10.5281/zenodo.10631439]{\doi{10.5281/zenodo.10631439}}. Given the complexity of the parameter space when
fitting multiple populations along with the recent warnings in the liteerature
regarding overfitting datasets \citep[e.g. ][]{Valle2022} we want to develop a
more objective way of fitting isochrones to photometry than if we were to mark
median ridge line positions by hand.

