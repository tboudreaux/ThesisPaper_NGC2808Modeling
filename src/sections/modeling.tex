\section{Stellar Models}\label{sec:modeling}
We use the Dartmouth Stellar Evolution Program \citep[DSEP, ][]{Dotter2008} to
generate stellar models. DSEP is a well-tested, one-dimensional stellar
evolution code which includes a mixing length model of convection,
gravitational settling, and diffusion.

We use DSEP to evolve stellar models ranging in mass from 0.3 to 2.0 solar
masses from the zero-age main sequence (ZAMS) to the tip of the red giant
branch. Below 0.7 $M_{\odot}$ we evolve a model every 0.03 $M_{\odot}$ and
above 0.7 $M_{\odot}$ we evolve a model every 0.5 $M_{\odot}$. Additionally, we
evolve models over a grid of mixing length parameters, $\alpha_{MLT}$, from
$\alpha_{MLT} = 1.0$ to $\alpha_{MLT} = 2.0$ in steps of 0.1. In addition to the
mixing length grid the evolved grid of models also has dimensions population (A
or E) and helium abundance ($Y$). Each model is evolved in DSEP's ``high
resolution'' mode and had a maximum allowed time step of 50 Myr. 

For each combination of population, $Y$, and $\alpha_{MLT}$ we use the
isochrone generation code first presented in \citet{Dotter2016} to generate a
grid of isochrones. The isochrone generation code identified equivalent
evolutionary points (EEPs) over a series of masses and interpolates between
them. {\color{blue} The grid of isochrones generated for this work is avalible
as a digital supplement to this paper.}

