\section{Modeling}\label{sec:modeling}
One key element of NGC 2808 modeling is the incorporation of new atmospheric
models, generated from the \texttt{MARCS} grid of model atmospheres \citep{Plez2008},
which match interior elemental abundances. \texttt{MARCS} provides one-dimensional,
hydrostatic, plane-parallel and spherical LTE atmopsheric models
\citep{Gustafsson2008}. Members of our collaboration have generated atmospheric 
models for populations A and E.  Integration of these new model atmospheres
into DSEP is ongoing. 

For similar reasons as discussed in \S\ref{sec:p1} we conduct this research
with OPLIB high-temperature opacity tables as opposed to OPAL tables. We will
also generate low temperature opacity tables using the \texttt{MARCS}.
Moreover, we confirm that the atmosphere and structure meet in an optically
thick region of the star by shifting the atmospheric fitting point from an optical
depth of $\tau = 2/3$ (used by DSEP currently for \texttt{PHOENIX} model atmospheres) to
some higher $\tau$. We will experiment to identify the best optical depth to
fit at.. 

These population have been studied in depth by Feiden and their chemical
compositions were determined in \citet{Milone2015} (see Table 2 in that paper).
While we cannot yet evolve DSEP models with these new boundary conditions, we
can make a first pass investigation of the affect of OPLIB opacities (Figure
\ref{fig:NGC2808ISO}). Note how the models generated using OPLIB opacity tables
have a systematically lower luminosity. This discrepancy is consistent with
the overall lower opacities of the OPLIB tables. 


The isochrones generally used to infer the degree of helium enhancements assume that
convection operates in the same manner in metal-poor stars as it does in the
Sun. However, observations from \textit{Kepler} of metal-poor red giants
\citep{Bonaca2012, tayar2017correlation}, in concert with interferometric
radius determination of the metal-poor sub-giant HD 140283
\citep{creevey2015benchmark}, have shown that the efficiency of convection
changes with iron content. As the final portion of our work to more carefully
handle a star's chemistry, we will modify DSEP to capture this variation in
convective efficiency. 
